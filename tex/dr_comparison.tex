\documentclass{article}
\usepackage{multicol}
\usepackage[utf8]{inputenc}
\usepackage{amsmath}
\usepackage[left=1in, right=1in, top=1in, bottom=1in]{geometry}
\usepackage{bbm}
\usepackage{mathtools}
\usepackage{amsfonts}
\usepackage{changepage}
\usepackage{amsthm}
\usepackage[linesnumbered,ruled,vlined]{algorithm2e}
\usepackage[backend=bibtex]{biblatex}
\theoremstyle{definition}
\newtheorem*{definition}{Definition}
\title{A Comprehensive Comparison of tSNE and UMAP}

\begin{document}
\maketitle

\begin{multicols}{2}

\begin{abstract}
Dimensionality reduction algorithms are used across scientific disciplines to visualize and understand data. Among them, tSNE and UMAP have become two of the most
popular due to their quick runtimes and intuitive results. Despite their ubiquity and similarities, however, there
has not been a comprehensive comparison of the two approaches. This work identifies a general framework for comparing dimensionality reduction algorithms and
analyzes tSNE and UMAP through the developed lens. We resolve misconceptions regarding components of each algorithm while highlighting theoretical and practical
similarities between them. In addition, we provide code that substantially accelerates both algorithms and provide theoretical insight for future
dimensionality reduction approaches.
\end{abstract}

\section{Dimensionality Reduction Algorithms}
We begin by formally introducing the tSNE and UMAP dimensionality reduction algorithms. Let $X \in \mathbb{R}^D$ be the high dimensional dataset of $N$ points and let $Y
\in \mathbb{R}^d$ be a randomly initialized set of $N$ points in some lower-dimensional space such that $d << D$. We establish a nearest neighbor graph among the
high-dimensional points where points $x_i$ and $x_j$ share an edge if either $x_i$ is $x_j$'s nearest neighbor \underline{or} $x_j$ is $x_i$'s nearest neighbor.

We now establish Gaussian and Student-t kernels on the respective high- and low-dimensional distances to represent the likelihood that $x_i$ chooses $x_j$ as
its nearest neighbor. These kernels are defined by
\begin{align}
    p^{tsne}_{j|i}(x_i, x_j) &= \dfrac{\text{exp}(-d(x_i, x_j)^2 / 2 \sigma_i^2)}{\sum_{k \neq l} \text{exp}(-d(x_k, x_l)^2 / 2 \sigma_k^2)} \\[0.5ex]
    q^{tsne}_{ij}(y_i, y_j) &= \dfrac{(1 + ||y_i - y_j||^2_2)^{-1}}{\sum_{k \neq l} (1 + ||y_k - y_l||^2_2)^{-1}} \\[1.5ex]
    p^{umap}_{j|i}(x_i, x_j) &= \dfrac{\text{exp} (-d(x_i, x_j)^2 + \rho_{i})}{\tau_i} \\[0.3ex]
    q^{umap}_{ij}(y_i, y_j) &= \left( 1 + a(||y_i - y_j||^2_2)^b \right) ^{-1}
\end{align}
where $d(x_i, x_j)$ is the high-dimensional distance function, $\sigma$ and $\tau$ are point-specific variance scalars, $\rho_i = \min_{j \neq i} d(x_i, x_j)$,
and $a$ and $b$ are constants. In practice, we can assume that $2 \sigma_i^2$ is functionally equivalent to $\tau_i$, and we will use $\tau$ when referring to
this kernel variance term.

The primary difference between the two is the change in normalization. tSNE normalizes the high- and low-dimensional relationships by all of the pairwise
distances whereas UMAP allows the Gaussian and Student-t kernels to remain untouched. We note that while the tSNE paper implies that high-dimensional
normalizations occur across rows of the pairwise distance matrix, the code performs normalization across the entire pairwise distance matrix. The high-dimensional kernels are symmetrized by applying
symmetrization functions. Without loss of generality, let $p_{ij} = S(p_{j|i}, p_{i|j})$.

Each algorithm then applies gradient descent with respect to the KL-divergence(s) between probability distributions. In the case of tSNE, the probability
distributions are defined by the matrices $P = \left( p_{ij} \right)_{i, j < N}$ and $Q = \left( q_{ij}
\right)_{i, j < N}$. However, UMAP's lack of normalization instead implies a single Bernoulli distribution for each edge. Thus, UMAP actually minimizes the sum of
KL divergences between every edge in high- and low-dimensional space. This gives us the loss functions
\begin{align}
    \mathcal{L}_{tsne} &= \text{KL} (P || Q) = \sum_{i \neq j} p_{ij} \log \dfrac{p_{ij}}{q_{ij}} \\
    \mathcal{L}_{umap} &= \sum_{i \neq j} \left[ p_{ij} \log \dfrac{p_{ij}}{q_{ij}} + (1 - p_{ij}) \log \dfrac{1 - p_{ij}}{1 - q_{ij}} \right]
\end{align}
where the operand inside $\mathcal{L}_{umap}$'s sum is the KL divergence of a Bernoulli distribution. We point out that usage of the KL divergence necessitates
symmetrical normalization between the high- and low-dimensional spaces, further supporting tSNE's implementation of the normalization over the asymmetrical
normalization discussed in the paper.

We can rearrange terms in $\mathcal{L}_{umap}$ to obtain
\[ \mathcal{L}_{umap} = \sum_{i \neq j} \left[ p_{ij} \log \dfrac{p_{ij}}{q_{ij}} \right] + \sum_{i \neq j} \left[ (1 - p_{ij}) \log \dfrac{1 - p_{ij}}{1 - q_{ij}} \right] \]
Notice that the first sum is identical to the sum in $\mathcal{L}_{tsne}$ up to normalization. Thus, the difference in normalization is directly accounted for
by the additional sum in $\mathcal{L}_{umap}$.

In practice, both tSNE and UMAP optimize this KL divergence by separately calculating attractive and repulsive forces. Thus, we can interpret the gradient
descent problem as a set of springs between every pair of points in $Y$ where the spring constants are determined by Gaussian and Student-t kernels.

The gradient of the KL divergences becomes substantially different due to the differing normalizations. In tSNE, the gradient can be written as
\begin{equation}
    \dfrac{\partial \mathcal{L}_{tsne}}{\partial y_i} = 4 \sum_{j \neq i} (p_{ij} - q_{ij}) q_{ij} Z (y_i - y_j)
\end{equation}
where $Z = \sum_{k \neq l} (1 + ||y_k - y_l||_2^2)^{-1}$ is the normalization factor for the low-dimensional kernel. This is often represented as an attractive
and repulsive force with
\begin{align*}
    \dfrac{\partial \mathcal{L}_{tsne}}{\partial y_i} &= 4(F^{tsne}_{attr} + F^{tsne}_{rep}) = \\
    &= 4 \left[ \sum_{j \neq i} p_{ij}q_{ij}Z (y_i - y_j) - \sum_{j \neq i} q_{ij}^2
Z (y_i - y_j) \right]
\end{align*}

UMAP also describes separating its gradient into attractive and repulsive terms, with
\begin{align}
    F_{attr}^{umap} = &\dfrac{-2ab||y_i - y_j||_2^{2(b-1)}}{1 + ||y_i - y_j||_2^2} p_{ij} (y_i - y_j) \\
    F_{rep}^{umap} = &\dfrac{2b}{(\epsilon + ||y_i - y_j||_2^2)(1 + a ||y_i - y_j||_2^{2b})} \cdot \\
    &\cdot (1 - p_{ij}) (y_i - y_j) \label{umap_rep}
\end{align}
However, we note that these are actually the gradients of the two terms in the summand of $\mathcal{L}_{umap}$, and each has its own attractive and repulsive
component within it. To avoid confusion, we will continue to refer to these as attractive and repulsive forces to stay consistent with the terminology used by
the original authors.

\section{Theoretical Analysis}
\subsection{A framework for dimensionality reduction}
We identify a general framework for studying dimensionality reduction algorithms that is based on the following observation: each dimensionality reduction
algorithm attempts to match kernel on high-dimensional distances with kernels on low-dimensional distances. This means that there are four defining
characteristics to each algorithm -- the high-dimensional kernel, the low-dimensional kernel, the similarity metric, and the optimization method. In the cases of tSNE and UMAP, these
are formally defined as exponential Gaussian kernels in the high-dimensional space and Student-t kernels in the low dimensional space. Both then use the
KL-divergence and gradient descent to optimize the embeddings.

\subsection{Implementation differences}
Despite using similar methods, there are several important differences in the implementation of these algorithms.

\begin{itemize}
    \item Similar to the kernel normalization, tSNE normalizes the repulsive forces by the sum of distances from every point to every other point in the
        dataset. In contrast, UMAP only applies repulsive forces with respect to a constant number of points. In practice, this means that the repulsive forces in UMAP are done per-point while the repulsive forces in tSNE are applied as an average across all of the points.

    \item tSNE collects all of the attractive and repulsive forces before applying momentum gradient descent across every point simultaneously. In contrast,
        UMAP updates the position of every point directly upon calculating their attractive and repulsive forces. This requires UMAP to apply $n$ repulsions for
        every attractive force. This also affects how each algorithm performs gradient clipping, where UMAP clips each gradient individually while tSNE clips
        the sum of the gradients across the entire epoch.

    \item UMAP finds approximate nearest neighbors using nearest-neighbor descent whereas tSNE takes the time to exactly identify nearest neighbor relationships.

    \item UMAP's high dimensional kernel on points $x_i$ and $x_j$ subtracts the minimum distance $\rho_i = \min_{k \neq i} d(x_i, x_k)$. There is a lot of
        work done to justify this choice in both the papers and presentations. 

    \item tSNE symmetrizes the high-dimensional kernels with $S_{tsne}(p_{j|i}, p_{i|j}) = (p_{j|i} + p_{i|j}) / 2$ while UMAP uses a probabilistic
        symmetrization with $S_{umap}(p_{j|i}, p_{i|j}) = p_{j|i} + p_{i|j} - p_{j|i} \cdot p_{i|j}$

    \item tSNE performs random initialization whereas UMAP initializes with a Laplacian eigenmap projection.

    \item tSNE applies the attractive force between $y_i$ and $y_j$ to only $y_i$'s position whereas UMAP applies the attractive force to both $y_i$ and $y_j$.
        We refer to these options as \textit{symmetric} and \textit{asymmetric attraction}.
\end{itemize}

\subsection{Uniform UMAP}
We propose a modified implementation of UMAP that we call Uniform UMAP. The principle difference consists in the gradient descent methodology. Each epoch of
UMAP loops through the nearest neighbor edges and performs one attraction along with $n$ repulsions. These gradient updates are performed live within the loop,
modifying the positions of each point as it is processed.

The first modification changes the gradient application methodology. Rather than updating positions within the loop, we can instead collect all of the
attractive and repulsive forces and apply them simultaneously at the end of the epoch. This in practice removes the need for calculating multiple repulsions
for every attraction, cutting down on the number of force calculations by $n-1$, where $n$ is the number of repulsions performed for every attraction. This is
complicated, however, by the fact that UMAP applies the $(1 - p_{ij})$ scaling in \ref{umap_rep} by sampling the edge proportionally to the weight. Due to
uniformly optimizing all of the edges, we cannot apply this sampling strategy. Instead, we assume that the randomly chosen pairwise repulsions will average out
and simply use the average weight $\bar{p}_{ij}$ as a substitute when calculating the UMAP repulsive forces. We theoretically substantiate this decision and
also experimentally validate that it does not impart a noticeable change in any of our test cases.

The second major modification has to do with the iterative nature of the gradient descent epochs. Consider the following common situation: assume during
epoch $t$ that we apply an attraction along the edge $(y_i, y_j)$ and then a repulsion along $(y_i, y_k)$. Then, in epoch $t+1$ we once again attract along
$(y_i, y_j)$ and then repel along $(y_i, y_l)$. Due to randomness, many individual repulsive forces are negligible. Thus, the previously described gradient
updates will oftentimes be approximately equal to performing the attraction $(y_i, y_j)$ at time $t$ twice and applying the two repulsive forces in tandem.
Furthermore, if we intend to apply an attractive force $k$ times then we can calculate apriori how much the force will diminish with each application. Thus, we can be
more sophisticated than simply scaling the force linearly by the number of repeated applications. \textit{It remains to identify when this approximation is appropriate
and to what extent it is correct}.

\section{Experimental Results}
We show that, subject to a few approximations, one can directly implement both tSNE and UMAP through the Uniform UMAP framework. We first show that Uniform UMAP
can successfully recreate UMAP's outputs across a variety of datasets. Then, we identify the changes necessary to produce tSNE's outputs. Finally, we end by
noting that implementing tSNE with UMAP's framework implies that one can implement UMAP within tSNE's framework. As such, we experimentally validate that
this is indeed possible.

\subsection{Achieving tSNE within UMAP}
The main obstacle to overcome is undoing the normalization
differences between tSNE and UMAP. In fact, tSNE can be achieved within the UMAP implementation through the following modifications:
\begin{enumerate}
    \item Normalize the high- and low-dimensional kernels as they are normalized in tSNE
    \item Apply asymmetric attraction forces
\end{enumerate}
Due to using tSNE's normalization, we also require the corresponding modification to the KL divergence. As such, we obtain gradients
\begin{align*}
    F_{attr} &= 4 \sum_{j \neq i} p_{ij} q_{ij} Z (y_i - y_j) \\
   F_{rep} &= 4 \sum_{j \neq i} q_{ij}^2 Z (y_i - y_j)
\end{align*}
   where
\begin{align*}
   q_{ij} &= \dfrac{(1 + a ||y_i - y_j||_2^{2b})^{-1}}{\sum_{k \neq l} (1 + a ||y_i - y_j||_2^{2b})^{-1}} \\
    p_{j|i} &= \dfrac{\text{exp}( (-d(x_i, x_j)^2 + \rho_i) / 2 \sigma_i^2)}{\sum_{k \neq l} \text{exp}( (-d(x_k, x_l)^2 + \rho_i) / 2 \sigma_k^2)}
\end{align*}
   and $Z = \sum_{k \neq l} (1 + a ||y_i
   - y_j||_2^{2b})^{-1}$. Notice that this is essentially the tSNE kernels with UMAP's $\rho_i$, $a$, and $b$ scalars. We show later that these can be removed
   in practice.

\subsection{Embedding analysis}
We now show multiple plots from applying multiple dimensionality reduction algorithms to several datasets. The plots are as follows:
\begin{enumerate}
    \item Change in sort index - If we sort the high-dimensional distances and the low-dimensional distances, how different are the sort indices? The x-axis
        goes through all the high-dimensional distances in order; the y-axis is the relative change for that high-dimensional sort index
    \item High dimensional distance vs. low dimensional distance - These are simply the low-dimensional distances plotted vs. the high dimensional distances. The high dimensional distances
        have been sorted from least to greatest. Note that PCA is a straight line (which is what we expected). Also, note that the large high-dim distances are
        represented by larger low-dim distances in UMAP than in tSNE
    \item Sorted distances - This is just the y-values of the previous plot. Basically, when we sort the high-dimensional distances, what do the low
        dimensional ones look like?
    \item Nearest neighbor overlap - For every point, what percent of its [k, k + 10] nearest neighbors are preserved? Note that immediate local distances are preserved in
        tSNE and UMAP, after which there is a giant portion where the middle distances aren't relevant.
    \item Relative error - This is the relative error for high-dimensional distances. We normalize both the high-dim and low-dim distances to $[0, 1]$, since the
        magnitudes are arbitrary in our setting. Thus, we do $| (d^h(x_i, x_j) - d^l(y_i - y_j)) |/ d^h(x_i, x_j)$. We replace the 0's in the high dimensional
        distances with 1's to avoid division errors.
\end{enumerate}

\section{Left to Show}
This is a scratchwork section for identifying remaining topics of study
\begin{itemize}
    \item tSNE and UMAP
        \begin{itemize}
            \item Should we show that the additional KL sum in UMAP directly accounts for the change in normalization?
        \end{itemize}
    \item Uniform UMAP
        \begin{itemize}
            \item When can we replace $k$ applications of an attractive force with a single application of a stronger attractive force
            \item Why does Uniform UMAP not need muliple repulsions for each attraction?
            \item GPU implementation of Uniform UMAP?
        \end{itemize}
    \item General
        \begin{itemize}
            \item Why do the PCA plots look the way that they do?
        \end{itemize}
\end{itemize}

\end{multicols}
\printbibliography
\end{document}
