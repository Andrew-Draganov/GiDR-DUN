\documentclass{article}
\usepackage[utf8]{inputenc}
\usepackage[left=1in, right=1in, top=1in, bottom=1in]{geometry}
\usepackage{amsmath}
\usepackage{bbm}
\usepackage{mathtools}
\usepackage{algorithm}
\usepackage{algpseudocode}
\usepackage{amsfonts}
\usepackage{changepage}
\usepackage{amsthm}
\usepackage{graphicx}
% \usepackage[linesnumbered,ruled,vlined]{algorithm2e}
\usepackage[backend=bibtex]{biblatex}
\theoremstyle{definition}
\newtheorem*{definition}{Definition}
\allowdisplaybreaks
\title{PCA as Gradient Descent}
\author{Andrew Draganov}

\begin{document}
\maketitle
\section{PCA objective function and gradient}

\begin{figure}
\centering
\includegraphics[width=.4\textwidth]{tex_images/0500.png}
\caption{The result of performing gradient descent PCA on the MNIST dataset}
\end{figure}

Given a high-dimensional dataset $X \in \mathbb{R}^{N \times D}$, we aim to identify a gradient descent objective to find points $Y \in \mathbb{R}^{N \times d}$
with $d << D$ such that $Y$ is the result of performing PCA on $X$.

We start with formalizing PCA as minimizing the function

\begin{equation}
    f_{PCA}(X, Y) = ||L (E^X - E^Y) L ||^2_F
\end{equation}
where $E^X \in \mathbb{R}^{N \times N}$ and $E^Y \in \mathbb{R}^{N \times N}$ are the pairwise squared distance matrix for $X$ and $Y$ and $L$ is the $N \times N$ centering matrix. Optimizing this function inherently
amounts to minimizing the Frobenius norm between the two matrices. We show two separate ways to calculate the gradient of $f_{PCA}$ with respect to the
low-dimensional points $Y$.

\subsection{Element-wise gradient calculations}

To identify the effect of gradient descent on $Y$, we can deconstruct the matrices into a set of element-wise operations. We can re-arrange the term in the
Frobenius norm:
\begin{align*}
    L (E^X - E^Y) L &= LE^XL - LE^YL \\
    &= E^X - \bar{X}^{\rightarrow} - \bar{X}^{\downarrow} - E^Y + \bar{Y}^{\rightarrow} + \bar{Y}^{\downarrow}
\end{align*}
where $\bar{X}^{\rightarrow}$ is the $N \times N$ matrix of row-means that the centering matrix on the right-hand-side of $X$ subtracts. The other $\bar{X}$ and
$\bar{Y}$ variables are defined accordingly. We can now rearrange these to obtain:
\[ L (E^X - E^Y) L = E^X - E^Y - \Lambda \]
for $\Lambda = - \bar{X}^{\rightarrow} - \bar{X}^{\downarrow} + \bar{Y}^{\rightarrow} + \bar{Y}^{\downarrow} \in \mathbb{R}^{N \times N}$

Then the PCA objective function can equivalently be written as
\begin{equation}
    f_{PCA}(X, Y) = \sum_{i, j} \left( ||x_i - x_j||^2 - ||y_i - y_j||^2 - \lambda_{ij} \right) ^2
\end{equation}

To perform gradient descent, we take the partial derivative with respect to $y_i$ to obtain
\[ \nabla_{y_i} f(X, Y) = -4 \sum_{j} (||x_i - x_j||^2 - ||y_i - y_j||^2 - \lambda_{ij}) (\dfrac{\partial \vec{\lambda}_{ij}}{\partial y_i} + \vec{y}_i - \vec{y}_j) \]

This gives us the following gradient descent algorithm with learning rate $\nu$:
\begin{equation*}
    \boxed{
    \begin{aligned}
        \nabla_{y_i} f(X, Y) &= -4 \sum_{j} (||x_i - x_j||^2 - ||y_i - y_j||^2 - \lambda_{ij}) \left( \dfrac{\partial \vec{\lambda}_{ij}}{\partial y_i}
        + \vec{y}_i - \vec{y}_j \right) \\
        Y_{t + 1} &= Y_t + \nu \nabla_{Y} f(X, Y)
    \end{aligned}
    }
\end{equation*}

This gives us the following algorithm:
\begin{algorithm}
\caption{PCA by gradient descent on the points}\label{alg:cap}
\begin{algorithmic}
\Require Input: $X \in \mathbb{R}^{N \times D}$, n\_epochs, $\nu$
\State $D^X \gets \text{pairwise\_dists}(X) \in \mathbb{R}^{N \times N}$
\State $D^X \gets D^X - \text{row\_means}(D^X)$
\State $D^X \gets D^X - \text{col\_means}(D^X)$
\State $Y \gets \mathcal{N}_{(0, 1)} \in \mathbb{R}^{N \times d}$
\While{$e < \text{n\_epochs}$}
\State $V^Y \gets \text{pairwise\_vectors}(Y) \in \mathbb{R}^{N \times N \times 2}$
\\
\State $D^Y \gets \text{pairwise\_dists}(Y) \in \mathbb{R}^{N \times N}$
\State $D^Y \gets D^Y - \text{row\_means}(D^Y)$
\State $D^Y \gets D^Y - \text{col\_means}(D^Y)$
\\
\State $ \nabla_Y \gets -4 \cdot \text{SUM}\left(( D^X - D^Y ) V^Y, \; \text{axis}=1 \right)$
\State $ Y \gets Y + \nu \nabla_Y$
\State $e++$
\EndWhile
\\
\Return $Y$
\end{algorithmic}
\end{algorithm}

Lastly, we note that we can separate the PCA gradient descent algorithm into a pair of alternating forces. For $\Lambda^X = \bar{X}^{\rightarrow}
+ \bar{X}^{\downarrow}$ and $\Lambda^Y = - \bar{Y}^{\rightarrow} - \bar{Y}^{\downarrow}$ Namely, we have
\begin{align*}
    \nabla_{y_i} f(X, Y) &= -4 \sum_{i, j} ( ||x_i - x_j||^2 - ||y_i - y_j||^2 - \lambda_{ij}) (\vec{y}_i - \vec{y}_j) \\
    &= 4 \sum_{i, j} (||x_i - x_j||^2 - \lambda^X_{ij}) (\vec{y}_i - \vec{y}_j) - 4 \sum_{i, j} (||y_i - y_j||^2 - \lambda^Y_{ij})(\vec{y}_i
    - \vec{y}_j) \\
    &= 4 \left( F_{X} - F_{Y} \right)
\end{align*}
Each force is dominated by the large and small distances. However, we can deconstruct these into clear attractive and repulsive forces. If points $x_i$
and $x_j$ are one another's nearest neighbors in the high-dimensional space, then $||x_i - x_j||^2$ will always be less than its respective centering
$\lambda^X_{ij}$. Thus, nearest neighbors in the high-dimensional space \textit{only} exert attractions on the embedding. Similarly, farthest
neighbors in the high-dimensional space \textit{only} create repulsive forces. Inversely, nearest neighbors in the embedding exert repulsions while
farthest neighbors exert attractions. PCA is the single convergence of these individual systems. Therefore, as one progresses from $x_i$'s nearest to its farthest neighbor, the force must this neighbor exerts on $x_i$ must change monotonically.


\subsection{Vectorized PCA gradient}
A different approach would be to rely on matrix differentiation using vectorizations and Kronecker products. We first establish the following preliminaries:
\begin{enumerate}
    \item Given Gram matrix $G_X = X^T X$, we can write $E^X$ as $E^X = \text{diag}(G_X) \mathbbm{1} + \mathbbm{1} \text{diag}(G_X)^T - 2G$ as the matrix of
        pairwise squared distances
    \begin{itemize}
        \item We simplify notation by writing $\text{diag}(G_X) = g_x$, giving us
            \[ E^X = g_X \mathbbm{1} + \mathbbm{1} g_X^T - 2G_X \]
        \item $\mathbbm{1}$ is the vector consisting of all 1's
    \end{itemize}
    \item The product of three matrices $ABC$ can be \textit{vectorized} (transformed into a column vector by linear transformation) using the operation
        $\textit{vec}(ABC) = (C^T \otimes A) \textit{vec}(B)$, where $\otimes$ represents the Kronecker product.
    \item Vectorized matrices are transposed by appropriately sized \textit{commutation matrices} $K$ s.t. for any matrix $A \in \mathbb{R}^{n \times m}$ it is
        the case that $K^{(m \times n)} \textit{vec}(A) = \textit{vec}(A^T)$, where $K \in \mathbb{R}^{mn \times mn}$
\end{enumerate}

We begin with the error function for PCA
\[ f_{PCA}(X, Y) = ||L(E^X - E^Y)L||_F^2 \]

By applying the first preliminary, we rearrange this by
\begin{align*}
    f_{PCA}(X, Y) &= ||L(E^X - E^Y)L||_F^2 \\
    f_{PCA}(X, Y) &= ||L(g_X \mathbbm{1} + \mathbbm{1} g_X^T - 2G_X - g_Y \mathbbm{1} - \mathbbm{1} g_Y^T + 2G_Y)L||_F^2 
\end{align*}

But now notice that $L \mathbbm{1} = 0 = \mathbbm{1}^T L$. So we can simplify this expression into
\[ f_{PCA}(X, Y) = ||L(2G_Y - 2G_X)L||_F^2 \]

The squared Frobenius norm $||M||_F^2$ can be rewritten as $tr(M^T M)$. Applying this gives us
\begin{align*}
    f_{PCA}(X, Y) &= tr \left( \left(L(2G_Y - 2G_X)L\right)^T L(2G_Y - 2G_X)L \right) \\
    &= 4 tr \left( L (G_Y - G_X)^T L L(G_Y - G_X)L \right) \\
    &= 4 tr \left( L (G_Y^T - G_X^T) L (G_Y - G_X)L \right) \\
    &= 4 tr \left( L (G_Y^T L G_Y - G_X^T L G_Y - G_Y^T L G_X + G_X^T L G_X )L \right) \\
\end{align*}

We now separate this by the linearity of the trace operation to obtain
\[ f_{PCA}(X, Y) = 4 \left[ tr \left( L G_Y^T L G_Y L \right) - tr \left( L G_X^T L G_Y L \right) - tr \left( L G_Y^T L G_X L \right) + tr \left( L G_X^T L G_X L \right) \right] \]

The product is commutative within the trace operation, so we can cancel the outer centering matrices to obtain
\begin{align*}
f_{PCA}(X, Y) &= 4 \left[ tr \left( G_Y^T L G_Y \right) - tr \left( G_X^T L G_Y \right) - tr \left( G_Y^T L G_X \right) + tr \left( G_X^T L G_X \right) \right]
\\
f_{PCA}(X, Y) &= 4 tr \left( G_Y^T L G_Y - G_X^T L G_Y - G_Y^T L G_X + G_X^T L G_X \right) \\
&\textit{By further commutative swaps and removing the transposes, we get} \rightarrow \\
\rightarrow f_{PCA}(X, Y) &= 4 tr \left( L \left( G_Y G_Y - 2 G_X G_Y + G_X G_X \right) \right) \\
&= 4 tr \left( L \left( G_Y - G_X \right)^2 \right)
\end{align*}

We now incorporate the fact that the trace of a product of symmetric matrices is the inner product of the vectorizations of those matrices
\[ \rightarrow f_{PCA}(X, Y) = 4 \langle \text{vec}( (G_Y - G_X)^2 ), \text{vec}( L ) \rangle \]

Taking the partial derivative with respect to $Y$ gives:
\begin{align*}
    d(f_{PCA}(X, Y)) &= 4\text{vec}(L)^T \text{vec}\left( d \left( (G_Y - G_X)^2 \right) \right) \\
    &= 4\text{vec}(L)^T \text{vec}\left( d \left( G_Y^2 \right) - d \left( G_Y G_X \right) - d \left( G_X G_Y \right) \right) \\
    &= 4\text{vec}(L)^T \text{vec}\left( d \left( G_Y^2 \right) - 2G_X \right) \\
    &= 4\text{vec}(L)^T \text{vec}\left( 2 G_Y d \left( G_Y \right) - 2G_X \right) \\
    &= 8\text{vec}(L)^T \text{vec}\left( G_Y (dY^T Y + Y^T dY) - G_X \right) \\
    &= 8\text{vec}(L)^T \text{vec}\left( G_Y (dY^T Y + Y^T dY)\right) - 8\text{vec}(L)^T \text{vec} \left( 2G_X \right) \\
    &= 8\text{vec}(L)^T \left( \text{vec} \left( G_Y dY^T Y \right) + \text{vec} \left( G_Y Y^T dY \right)\right) - 4\text{vec}(L)^T \text{vec} \left( 2G_X \right) \\
\end{align*}

We now extract the $dY$ vectors by performing the $\text{vec}(ABC) = (C^T \otimes A) \text{vec}(B)$ trick:
\begin{align*}
    df_{PCA}(X, Y) &= 8\text{vec}(L)^T \left( (Y^T \otimes G_Y)K dY + (I \otimes G_Y Y^T) dY \right) - 4\text{vec}(L)^T \text{vec} \left( 2G_X \right) \\
    &= 8\text{vec}(L)^T \left[ \left( (Y^T \otimes G_Y)K + (I \otimes G_Y Y^T) \right) dY \right] - 4\text{vec}(L)^T \text{vec} \left( 2G_X \right) \\
\end{align*}

\end{document}
